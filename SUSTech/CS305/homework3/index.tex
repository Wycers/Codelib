\documentclass[12pt,letterpaper]{ctexart}
\usepackage{fullpage}
\usepackage[top=2cm, bottom=4.5cm, left=2.5cm, right=2.5cm]{geometry}
\usepackage{amsmath,amsthm,amsfonts,amssymb,amscd}
\usepackage{lastpage}
\usepackage{enumerate}
\usepackage[binary-units=true]{siunitx}
\usepackage{fancyhdr}
\usepackage{mathrsfs}
\usepackage{xcolor}
\usepackage{graphicx} %插入图片的宏包
\usepackage{float} %设置图片浮动位置的宏包
\usepackage{subfigure} %插入多图时用子图显示的宏包
\usepackage{listings}
\usepackage{afterpage}
\usepackage{hyperref}
\hypersetup{
    colorlinks=true,
    linkcolor=blue,
    filecolor=magenta,
    urlcolor=cyan,
}

\newcommand\blankpage{%
  \null
  \thispagestyle{empty}%
  \addtocounter{page}{-1}%
  \newpage
}


\hypersetup{%
  colorlinks=true,
  linkcolor=blue,
  linkbordercolor={0 0 1}
}

\renewcommand\lstlistingname{Algorithm}
\renewcommand\lstlistlistingname{Algorithms}
\def\lstlistingautorefname{Alg.}

\lstdefinestyle{Python}{
    language        = Python,
    frame           = lines,
    basicstyle      = \footnotesize,
    keywordstyle    = \color{blue},
    stringstyle     = \color{green},
    commentstyle    = \color{red}\ttfamily
}

\setlength{\parindent}{0.0in}
\setlength{\parskip}{0.05in}

% Edit these as appropriate
\newcommand\course{CS305}
\newcommand\hwnumber{3}                  % <-- homework number
\newcommand\NetIDa{11711918}           % <-- NetID of person #1
\newcommand\NetIDb{吴烨昌}           % <-- NetID of person #2 (Comment this line out for problem sets)

\pagestyle{fancyplain}
\headheight 35pt
\lhead{\NetIDa}
\lhead{\NetIDa\\\NetIDb}                 % <-- Comment this line out for problem sets (make sure you are person #1)
\chead{\textbf{\Large Homework \hwnumber}}
\rhead{\course \\ \today}
\lfoot{}
\cfoot{}
\rfoot{\small\thepage}
\headsep 1.5em

\begin{document}

\section*{Problem 1}

{\bf Description}

Consider a datagram network using 8-bit host addresses. Suppose a router uses longest prefix matching and has the following forwarding table:

\begin{table}[htbp]
  \small
  \centering
  \begin{tabular}{|c|c|}
  \hline
  Prefix & Match \\ \hline
  00     & 0     \\ \hline
  01     & 1     \\ \hline
  10     & 1     \\ \hline
  110    & 2     \\ \hline
  111    & 3     \\ \hline
  \end{tabular}
\end{table}
{\bf Solution}

\begin{table}[htbp]
  \small
  \centering
  \begin{tabular}{|c|c|c|}
  \hline
  Interface & Range of destination                                  & Number of addresses  \\ \hline
  0         & 0b00000000$\sim$0b00111111                            & 64    \\ \hline
  1         & 0b10000000$\sim$0b10111111,0b01000000$\sim$0b01111111 & 128   \\ \hline
  2         & 0b11000000$\sim$0b11011111                            & 32    \\ \hline
  3         & 0b11100000$\sim$0b11111111                            & 32    \\ \hline
  \end{tabular}
\end{table}

\newpage

\section*{Problem 2}

{\bf Description}

Consider a router that interconnects three subnets: Subnet 1, Subnet 2, and Subnet 3.
Suppose all interfaces in these three subnets are required to have the prefix 222.1.16/24. Also
suppose that Subnet 1 is required to support at least 60 interfaces, Subnet 2 is to support at least
90 interfaces, and Subnet 3 is to support at least 12 interfaces. Provide three network addresses
(of the form a.b.c.d/x) that satisfy these constraints, please show the calculation procedure.

{\bf Solution}

Subnet 1 needs at least 60 interfaces, which means that subnet 1 needs at least 62 addresses. Because $2^6 >= 62$, network address of subnet 1 is \textbf{222.1.16.128/26}.

Subnet 2 needs at least 90 interfaces, which means that subnet 1 needs at least 92 addresses. Because $2^7 >= 92$, network address of subnet 2 is \textbf{222.1.16.0/25}.

Subnet 3 needs at least 12 interfaces, which means that subnet 1 needs at least 14 addresses. Because $2^4 >= 14$, network address of subnet 3 is \textbf{222.1.16.192/28}.

\newpage

\section*{Problem 3}

{\bf Description}

Consider the network setup in the figure below. Suppose that the ISP instead assigns the
router the address 24.34.112.232 and that the network address of the home network is
192.168.2.0/25.

\begin{enumerate}
  \item Assign addresses to all interfaces in the home network.
  \item Suppose each host has two ongoing TCP connections, all to port 80 at host 128.119.40.87.
\end{enumerate}

Provide the six corresponding entries in the NAT translation table.

{\bf Solution}

\begin{enumerate}
  \item The router interfaces assigned as \textbf{192.168.2.4}. The rest assigned as follows:
  \begin{enumerate}
    \item \textbf{192.168.2.1}
    \item \textbf{192.168.2.2}
    \item \textbf{192.168.2.3}
  \end{enumerate}
  \item As follows
  \begin{table}[htbp]
    \small
    \centering
    \begin{tabular}{|c|c|c|}
    \hline
        WAN Side        &      LAN Side     \\ \hline
    24.34.112.232, 1145 & 192.168.2.1, 2333 \\ \hline
    24.34.112.232, 1141 & 192.168.2.1, 23333 \\ \hline
    24.34.112.232, 1144 & 192.168.2.2, 2333 \\ \hline
    24.34.112.232, 1149 & 192.168.2.2, 23333 \\ \hline
    24.34.112.232, 1148 & 192.168.2.3, 2333 \\ \hline
    24.34.112.232, 1140 & 192.168.2.3, 23333 \\ \hline
    \end{tabular}
  \end{table}


\end{enumerate}
\newpage

\section*{Problem 4}
{\bf Description}

What is the difference between a forwarding table that we encountered in destinationbased forwarding in Section 4.1 and OpenFlow's flow table that we encountered in Section 4.4?

{\bf Solution}

A forwarding table uses IP address or MAC addresses to decide the next hop for the packet.

A flow table may use any of the information within the packet to decide the next hop for it.


\section*{Problem 5}
{\bf Description}

Consider the following network. With the indicated link costs, use Dijkstra’s shortest-path
algorithm to compute the shortest path from x to all network nodes. Show how the algorithm
works and show the final forwarding table in x.

{\bf Solution}

\begin{table}[htbp]
  \small
  \centering
  \begin{tabular}{|c|c|c|c|c|c|c|}
  \hline
   x & y & z & w & u & v & t \\ \hline
   0 & * & * & * & * & * & * \\ \hline
   0 & 6 & 8 & 6 & * & 2 & * \\ \hline
   0 &   &   &   & 5 &   & 6 \\ \hline
  \end{tabular}
\end{table}

Final forwarding table is

\begin{table}[htbp]
  \small
  \centering
  \begin{tabular}{|c|c|c|c|c|c|c|}
  \hline
  Destination & Link \\ \hline
       t      &  (x, v)  \\ \hline
       u      &  (x, v)  \\ \hline
       v      &  (x, v)  \\ \hline
       w      &  (x, w)  \\ \hline
       y      &  (x, y)  \\ \hline
       z      &  (x, z)  \\ \hline
  \end{tabular}
\end{table}


\section*{Problem 6}
Consider the network fragment shown below. x has only two attached neighbors, w and y. w
has a minimum-cost path to destination u (not shown) of 5, and y has a minimum-cost path to u
of 6. The complete paths from w and y to u (and between w and y) are not shown. All link costs
in the network have strictly positive integer values.
\begin{enumerate}
  \item Give x’s distance vector for destinations w, y, and u.
  \item Give a link-cost change for either c(x,w) or c(x,y) such that x will inform its neighbors of a new
  minimum-cost path to u as a result of executing the distance-vector algorithm.
  \item Give a link-cost change for either c(x,w) or c(x,y) such that x will not inform its neighbors of a
  new minimum-cost path to u as a result of executing the distance-vector algorithm.
\end{enumerate}

{\bf Solution}

\begin{enumerate}
  \item As follows:
  \begin{table}[htbp]
    \small
    \centering
    \begin{tabular}{|c|c|c|c|c|c|c|}
    \hline
    Distance-vector  & Value \\ \hline
         c(x, w)     &  2    \\ \hline
         c(x, y)     &  $min(2 + 2, 5) = 4$ \\ \hline
         c(x, u)     &  $min(2 + 5, 4 + 6) = 7$  \\ \hline
    \end{tabular}
  \end{table}
  \item set $c(x, w)$ a value $\geq 7$, such as $7$.
  \item set $c(x, y)$ a value $\geq 1$, such as $2$.
\end{enumerate}

\end{document}
