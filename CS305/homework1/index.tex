\documentclass[12pt,letterpaper]{ctexart}
\usepackage{fullpage}
\usepackage[top=2cm, bottom=4.5cm, left=2.5cm, right=2.5cm]{geometry}
\usepackage{amsmath,amsthm,amsfonts,amssymb,amscd}
\usepackage{lastpage}
\usepackage{enumerate}
\usepackage[binary-units=true]{siunitx}
\usepackage{fancyhdr}
\usepackage{mathrsfs}
\usepackage{xcolor}
\usepackage{graphicx}
\usepackage{listings}
\usepackage{hyperref}
\usepackage{afterpage}

\newcommand\blankpage{%
  \null
  \thispagestyle{empty}%
  \addtocounter{page}{-1}%
  \newpage
}


\hypersetup{%
  colorlinks=true,
  linkcolor=blue,
  linkbordercolor={0 0 1}
}

\renewcommand\lstlistingname{Algorithm}
\renewcommand\lstlistlistingname{Algorithms}
\def\lstlistingautorefname{Alg.}

\lstdefinestyle{Python}{
    language        = Python,
    frame           = lines,
    basicstyle      = \footnotesize,
    keywordstyle    = \color{blue},
    stringstyle     = \color{green},
    commentstyle    = \color{red}\ttfamily
}

\setlength{\parindent}{0.0in}
\setlength{\parskip}{0.05in}

% Edit these as appropriate
\newcommand\course{CS305}
\newcommand\hwnumber{1}                  % <-- homework number
\newcommand\NetIDa{11711918}           % <-- NetID of person #1
\newcommand\NetIDb{吴烨昌}           % <-- NetID of person #2 (Comment this line out for problem sets)

\pagestyle{fancyplain}
\headheight 35pt
\lhead{\NetIDa}
\lhead{\NetIDa\\\NetIDb}                 % <-- Comment this line out for problem sets (make sure you are person #1)
\chead{\textbf{\Large Homework \hwnumber}}
\rhead{\course \\ \today}
\lfoot{}
\cfoot{}
\rfoot{\small\thepage}
\headsep 1.5em

\begin{document}

\section*{Problem 1}

{\bf Description}

Compare packet switch and circuit switch under the following scenario.
Suppose you would like to deliver a message of $x$ bit.
There are $k$ links from the source to destination.
The propagation delay of each link is $d$ second, the transmission rate is $b$ bit/second.
The circuit setup time under circuit switch is $s$ second.
Under packet switch network, when the packet length is $p$ bit, the queue delay in every node can be neglected.
Please calculate the condition, under which the delay of packet switch is smaller than that of the circuit switch.

{\bf Solution}

  In circuit switch, total delay is

  $$
  t_{c} = s + k \times d + \frac{x}{b}
  $$

  In packet switch, total delay is

  $$
  t_{s} = \frac{(k - 1) \times p }{b} + \lceil \frac{x}{p}\rceil \times \frac{p}{b} + k \times  d
  $$

  When $t_s < t_c$, or $ (k - 1) \times p  + \lceil \frac{x}{p}\rceil \times p < s \times b + x $, the delay of packet switch is smaller than that of the circuit switch.

%   \item
%     Problem 1 part 2 answer here.

%     Here is an example typesetting mathematics in \LaTeX
% \begin{equation*}
%     X(m,n) = \left\{\begin{array}{lr}
%         x(n), & \text{for } 0\leq n\leq 1\\
%         \frac{x(n-1)}{2}, & \text{for } 0\leq n\leq 1\\
%         \log_2 \left\lceil n \right\rceil \qquad & \text{for } 0\leq n\leq 1
%         \end{array}\right\} = xy
% \end{equation*}

%     \item Problem 1 part 3 answer here.

%     Here is an example of how you can typeset algorithms.
%     There are many packages to do this in \LaTeX.

%     \lstset{caption={Caption for code}}
%     \lstset{label={lst:alg1}}
%      \begin{lstlisting}[style = Python]
%     from package import Class # Mesh required for..

%     cinstance = Class.from_obj('class.obj')
%     cinstance.go()
%     \end{lstlisting}

%   \item Problem 1 part 4 answer here.

%     Here is an example of how you can insert a figure.
%     \begin{figure}[!h]
%     \centering
%     % \includegraphics[width=0.3\linewidth]{heidi.jpg}
%     \caption{Heidi attacked by a string.}
%     \end{figure}

% \afterpage{\blankpage}
\newpage

\section*{Problem 2}

{\bf Description}

Calculate the overall delay of transmitting a \SI{1000}{\kilo\byte} file under the following circumstance.
The overall delay is defined as the time from the starting point of the transmission until the arrival of the last bit to the destination.
RTT is assumed to be \SI{100}{\ms}, one packet is \SI{1}{\kilo\byte} (\SI{1024}{\byte}) size.
The handshaking process costs \SI{2}{RTT} before transmitting the file.
% Rest of the work...
\begin{enumerate}
  \item Transmission bandwidth is \SI[per-mode=symbol]{1.5}{\mega\byte\per\second}, the packets can be continuously transmitted.
  \item Transmission bandwidth is \SI[per-mode=symbol]{1.5}{\mega\byte\per\second}, but when one packet is transmitted, the next packet should wait for 1 RTT (waiting for the acknowledgement of the receiver) before being transmitted.
  \item Transmission bandwidth is infinite, i.e. transmission delay is 0. After every \SI{1}{RTT}, as many as 20 packets can be transmitted.
\end{enumerate}

{\bf Solution}

The number of packets $n$ is $\frac{1000\text{kB}}{1\text{kB}}=1000$.

\begin{enumerate}
  \item The delay under {\bf circumstance 1}

  $$
  t_1 = 2\text{RTT} + n \times \frac{1\text{kB}}{1.5\times1024\text{kB/s}} + 0.5\text{RTT} \approx 0.901\text{s}
  $$

  \item The delay under {\bf circumstance 2}

  $$
  t_2 = 2\text{RTT} + n \times (\frac{1\text{KB}}{1.5\times1024\text{KB/s}} + 0.5\text{RTT}) - 0.5\text{RTT} = t_1 + 0.5(n - 1) \text{RTT} \approx 50.851s
  $$

  \item The delay under {\bf circumstance 3}

  $$
  t_3 = 2\text{RTT} + \frac{n}{20/\text{RTT}} = 52\text{RTT} = 5.2\text{s}
  $$
\end{enumerate}

\section*{Problem 3}

{\bf Description}

List six access technologies. Classify each of them as home access, enterprise access, or wide-area mobile access.

{\bf Solution}

\begin{itemize}
  \item Home access: Dial-up modem over telephone line, Hybrid fiber-coaxial cable, WLAN
  \item Enterprise access: IEEE 802.1X
  \item Wide-area mobile access: 3G, 4G, 5G
\end{itemize}

\section*{Problem 4}
{\bf Description}

\begin{enumerate}
  \item List five nonproprietary Internet applications and the application-layer protocols that they use.
  \item What information is used by a process running on one host to identify a process running on another host?
\end{enumerate}

{\bf Solution}

\begin{enumerate}
  \item As followed
  \begin{enumerate}
    \item  web: http/https
    \item e-mail: imap/smtp/pop3
    \item remote desktop: rdp
    \item file transform: ftp
    \item remote access: ssh, telnet
  \end{enumerate}
  \item The IP address and port number of the destination host
\end{enumerate}

\newpage

\chead{\textbf{\Large Lab 3}}
\section*{Problem 1}

\end{document}
