\documentclass[12pt,letterpaper]{ctexart}
\usepackage{fullpage}
\usepackage[top=2cm, bottom=4.5cm, left=2.5cm, right=2.5cm]{geometry}
\usepackage{amsmath,amsthm,amsfonts,amssymb,amscd}
\usepackage{lastpage}
\usepackage{enumerate}
\usepackage[binary-units=true]{siunitx}
\usepackage{fancyhdr}
\usepackage{mathrsfs}
\usepackage{xcolor}
\usepackage{graphicx} %插入图片的宏包
\usepackage{float} %设置图片浮动位置的宏包
\usepackage{subfigure} %插入多图时用子图显示的宏包
\usepackage{listings}
\usepackage{afterpage}
\usepackage{hyperref}
\hypersetup{
    colorlinks=true,
    linkcolor=blue,
    filecolor=magenta,
    urlcolor=cyan,
}

\newcommand\blankpage{%
  \null
  \thispagestyle{empty}%
  \addtocounter{page}{-1}%
  \newpage
}


\hypersetup{%
  colorlinks=true,
  linkcolor=blue,
  linkbordercolor={0 0 1}
}

\renewcommand\lstlistingname{Algorithm}
\renewcommand\lstlistlistingname{Algorithms}
\def\lstlistingautorefname{Alg.}

\lstdefinestyle{Python}{
    language        = Python,
    frame           = lines,
    basicstyle      = \footnotesize,
    keywordstyle    = \color{blue},
    stringstyle     = \color{green},
    commentstyle    = \color{red}\ttfamily
}

\setlength{\parindent}{0.0in}
\setlength{\parskip}{0.05in}

% Edit these as appropriate
\newcommand\course{CS305}
\newcommand\hwnumber{5}                  % <-- homework number
\newcommand\NetIDa{11711918}           % <-- NetID of person #1
\newcommand\NetIDb{吴烨昌}           % <-- NetID of person #2 (Comment this line out for problem sets)

\pagestyle{fancyplain}
\headheight 35pt
\lhead{\NetIDa}
\lhead{\NetIDa\\\NetIDb}                 % <-- Comment this line out for problem sets (make sure you are person #1)
\chead{\textbf{\Large Assignment \hwnumber}}
\rhead{\course \\ \today}
\lfoot{}
\cfoot{}
\rfoot{\small\thepage}
\headsep 1.5em

\begin{document}

\section*{Problem 1}

{\bf Description}

\begin{itemize}
  \item make an DNS query which will invoke the EDNS0
  \begin{itemize}
    \item Screenshot on this command and its output
  \end{itemize}
  \item capture the packages using Wireshark
  \begin{itemize}
    \item what is the content of this query message
    \begin{itemize}
      \item Find the name, type and class of this query
      \item How can you tell this DNS query is based on EDNS0
      \item From this query massage , can it handle DNSSEC security RRs or not
    \end{itemize}
    \item what is the content of this response message
    \begin{itemize}
      \item Is there any answers, what’s the ttl of each answer
      \item Is there any authority RRs, what’s the type of each RR
      \item Is there any special additional RRs with OPT type, what does its ‘Do bit’ say: Does it accept DNSSEC security RRs or not?
    \end{itemize}
  \end{itemize}
\end{itemize}


{\bf Solution}

\begin{itemize}
  \item make an DNS query which will invoke the EDNS0
  \begin{itemize}
    \item Screenshot(Figure~\ref{fig:dig}) on this command and its output
    \begin{figure}[H]
      \centering
      \includegraphics[width=0.8\linewidth]{assets/dig.png}
      \caption{dig output}
      \label{fig:dig}
    \end{figure}
  \end{itemize}

  \item capture the packages using Wireshark
  \begin{itemize}
    \item what is the content of this query message
    \begin{itemize}
      \item Name: baidu.com, Type: A, Class of query: IN.
      \item From Figure~\ref{fig:edns0}, this DNS query is based on EDNS0.
      \begin{figure}[H]
        \centering
        \includegraphics[width=0.8\linewidth]{assets/edns0.png}
        \caption{Query based on EDNS0}
        \label{fig:edns0}
      \end{figure}
      \item From Figure~\ref{fig:secure}, it cannot handle DNSSEC security RRs.
      \begin{figure}[H]
        \centering
        \includegraphics[width=0.8\linewidth]{assets/secure.png}
        \caption{Cannot handle DNSSEC security RRs}
        \label{fig:secure}
      \end{figure}
    \end{itemize}
    \item what is the content of this response message
    \begin{itemize}
      \item From Figure~\ref{fig:ttl}, there are answers, TTL is 114.
      \begin{figure}[H]
        \centering
        \includegraphics[width=0.8\linewidth]{assets/ttl.png}
        \caption{TTL is 114}
        \label{fig:ttl}
      \end{figure}
      \item  From Figure~\ref{fig:authorityRRs}, there are 5 authority RRs, all are NS records.
      \begin{figure}[H]
        \centering
        \includegraphics[width=0.8\linewidth]{assets/authorityRRs.png}
        \caption{5 authority NS RRs}
        \label{fig:authorityRRs}
      \end{figure}
      \item There is special additional RRs with OPT type. Its `Do bit' says ``Cannot handle DNSSEC security RRs''.
      It does not accept DNSSEC security RRs.
      \begin{figure}[H]
        \centering
        \includegraphics[width=0.8\linewidth]{assets/additionalRRs.png}
        \caption{5 authority NS RRs}
        \label{fig:additionalRRs}
      \end{figure}
    \end{itemize}
  \end{itemize}
\end{itemize}

\newpage

\section*{Problem 2}

{\bf Description}

\begin{itemize}
  \item Make the query by using query method of ``dns resolver''(a python package)
  \begin{itemize}
    \item To query the type A value of \href{www.sina.com.cn}{www.sina.com.cn} based on TCP and UDP stream respectively
  \end{itemize}
  \item capture the related TCP stream and UDP stream using Wireshark
  \begin{itemize}
    \item Screenshot on this two commands.

    what's the default transport lay protocol while invoke DNS query
    \item Screenshot on the TCP stream of query by TCP.

    how many TCP packets are captured in this stream, Which port is used?
    \item Screenshot on the UDP stream of query by UDP

    how many UDP packets are captured in this stream, Which port is used?
    \item Is there any difference on DNS query and response message while using TCP and UDP respectively
  \end{itemize}
\end{itemize}


{\bf Solution}

\begin{itemize}
  \item Capture the related TCP stream and UDP stream using Wireshark
  \begin{itemize}
    \item Screenshot on this two commands.
    \begin{figure}[H]
      \centering
      \includegraphics[width=0.8\linewidth]{assets/cmd_udp.png}
      \caption{Command to send DNS query by UDP}
      \label{fig:cmd_udp}
    \end{figure}
    \begin{figure}[H]
      \centering
      \includegraphics[width=0.8\linewidth]{assets/cmd_tcp.png}
      \caption{Command to send DNS query by TCP}
      \label{fig:cmd_tcp}
    \end{figure}

    The default transport lay protocol while invoke DNS query is {\bf UDP}.
    \item Screenshot on the TCP stream of query by TCP.

    \begin{figure}[H]
      \centering
      \includegraphics[width=0.8\linewidth]{assets/pic_tcp.png}
      \caption{TCP stream of query by TCP.}
      \label{fig:pic_tcp}
    \end{figure}
    13 TCP packets are captured(DNS Protocol is on TCP), port 50980 is used.
    \item Screenshot on the UDP stream of query by UDP

    \begin{figure}[H]
      \centering
      \includegraphics[width=0.8\linewidth]{assets/pic_udp.png}
      \caption{UDP stream of query by UDP.}
      \label{fig:pic_udp}
    \end{figure}
    3 UDP packets(datagrams?) are captured in this stream, port 63680 is used.
    \item The messages(DNS results) are exactly the same. The only difference between them are length, port and header.
  \end{itemize}
\end{itemize}

\newpage

\section*{Problem 3}

{\bf Description}

\begin{itemize}
  \item Function:
  \begin{itemize}
    \item Listen and accept DNS queries. Support common query types: A, AAAA, CNAME, TXT, NS, MX
    EDNS implementation is not required.
    \item Forward query to a upstream DNS resolver (or a public DNS server).
    \item Check out the response and send response to your clients.
    \item Maintain a cache of DNS query-response of all results.
  \end{itemize}
  \item Test method:
  \begin{itemize}
    \item using dig sending query to your resolver
  \end{itemize}
\end{itemize}

{\bf Solution}

See 5.3.zip/dns\_resolver.py. Serve on 127.0.0.1 and port 5300
\end{document}
